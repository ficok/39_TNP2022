\documentclass{beamer}
\usepackage{beamerthemeshadow}
\usepackage{graphicx}
\usepackage{color}
\usepackage[utf8]{inputenc}
\usepackage[T2A]{fontenc}
\usepackage{hyperref}
\usepackage[flushleft]{threeparttable}
\definecolor{beamer@darkred}{rgb}{0.1,0.85,0.4}
\setbeamercolor{structure}{fg=beamer@darkred}





%литература и насловна страна иду изнад
%када пишете имаћемо два section-a: Filipov i moj deo, dok Isidora i Nikola kada budu pisali svoj treba da koriste \subsection, razlog jer po pravilu u pregledu ne bi trebali da imamo vise od 3 glavne teme tj. sectiona
\begin{document}
	
\begin{frame}
	\frametitle{Преглед}
	\tableofcontents
\end{frame}

%сви остали делови иду изнад мог
\section{Мере заштите}

\begin{frame}[fragile]\frametitle{Мере заштите}
	\begin{itemize}	
	\item Мере заштите које могу применити појединци
		\begin{itemize}
		\item опрезност на интернету
		\item редовно ажурирање оперативног система и антивирусног софтвера
		\item коришћење комплексних лозинки
		\item креирање резервних копија података
	\end{itemize}
	\item Мере заштите које могу примењивати организације
	\begin{itemize}
		\item спровођење редовних обука из домена сајбер безбедности 
		\item креирање профила за приступ
		\item креирање процедура за поступање у случајевима напада
		\item употреба анализе рањивости
		\item коришћењем мамаца ханипотова
	\end{itemize}
	\end{itemize}
\end{frame}
%verovatno cu dodati jos jedan slajd kao zakljucak
\end{document}